\section{Assumptions}
\label{sec:assumptions}
\autoref{fig:events} shows how complex even the initial events during a
collision are. As it would be too difficult to analyse these subsequent
situations without the proper software, only the first event (A) is
taken into account to reduce the complexity. To decrease the complexity even
more, a number of assumptions are made. In the following paragraphs these
assumptions are explained in detail.

Our victim is a healthy thirty year old male with a height of 1,80m and a mass
of 80kg. Before the collision he is located right in front of the car.
He is walking and at the moment of the crash the leg closest to the car is in
stand phase. The other leg is in swing phase. We examine the femur of the former
leg, because it is most likely to fail. The weight of both his upper and lower
leg segments are also calculated. It is assumed the weight is equally
distributed over the length of both parts, so the center of gravity is situated
in the middle of both parts. Using \autoref{fig:proportions}, we estimate the
distance from respectively the angle, knee and hip to the ground to be 0,07m,
0,51m and 0,95m. The lower leg has a mass of 3,72kg and the upper leg has a mass
of 8kg\cite{Ob}.  \cite{huiskes1977geometrical}
suggest $E_{femur}$ = 20 GPa and $\nu_{femur}$ = 0,37.

\begin{figure}[htp]
\begin{center}
  \includegraphics{img/proportions.png}
  \caption{Various lengths of human body segments in proportion to
  total height.}
  \label{fig:proportions}
\end{center}
\end{figure}

The car weighs 2400kg,  the bull bar is made of aluminium ($E_{Al}$=69 GPa and
$\nu_{Al}$=0,32)
% stainless steel ($E_{SS}$=210 GPa and $\nu_{SS}$=0,305
% \footnote{\url{http://www.engineeringtoolbox.com/poissons-ratio-d_1224.html}})
and its most protuding part sits at 0,80m heigh. At the time of collision the
car is decelerating, but it still has a speed of 36 km/h (10 m/s). Due to the
bullbar that is mounted on the bumper, the initial contact between the body of
the pedestrian and the car is limited to the bar touching the upper leg of the
pedestrian. The femur is the long bone that provides support in the upper leg.
Around the femur the hamstring muscles, the quadriceps muscles and a layer of
fat act as shock absorbers during the impact. The literature suggests using a
damping factor of roughly 15\% \cite{kannus1999comparison}.

As only the initial phase of the impact is considered in this paper, it is
assumed that the horizontal velocity of the hip is zero at the time of impact
due to the inertia of the body. This boundary condition is also used in the
literature \cite{snedeker2005assessing}. The foot also remains in place, but
rotates inwards at the ankle. Because of this the bones in the lower leg do not
deform. Visual analysis of crash dummy tests
\footnote{\url{https://www.youtube.com/watch?v=tNRHB75NiIc}} confirms this.
Further it is assumed that the knee does not bend initially. So the deformation
induced by the bull bar is only reflected in the femur shaft. In reality the
knee will also be affected by the forces exerted by the bull bar.
Depending on the situation and the relative strength of both, the femur will
break or the knee will be distorted. In the following analysis however, it is
assumed that the femur will absorb the initial impact. This means we assume that
the knee will not become distorted (so it will continue to properly connect the
upper and lower leg) and that the other bones -- from pelvis to the little bones
in the foot -- are not affected initially.

Because of these (rather rough) assumptions the whole leg can be treated as one
structure which is clamped at the top (pelvis) and which hinges at the level of
the ankle. The force exerted on this structure by the bull bar is treated as a
force concentrated in one point at a height of 80 cm. It is assumed that the
bull bar will not deform during the collision, nor that the crumple zone of the
car is activated. The complete system is visualised in \autoref{fig:hyper1}.

 \begin{figure}[htp]
\begin{center}
  \includegraphics[page=1,width=\textwidth]{img/hyper.pdf}
  \caption{Schematic of the system under consideration. The pelvis (left) is
  modeled as a clamp, while the angle (right) is replaced by a hinge. The knee
  (circle) is merely added for visual guidance, it does not play any
  significant part in this analysis. The total length $L$ is split up in $L_1$
  and $L_2$ at the point of impact.}
  \label{fig:hyper1}
\end{center}
\end{figure}

As the pedestrian is a young male it is assumed he has normal bones, not
affected by any disease. The femoral bone has an average length of 48 cm, a
shaft diameter of 2.43 cm and the femoral canal has a diameter of 13
mm.\footnote{\url{http://www.orthopaedicsone.com}}
