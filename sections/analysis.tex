\section{Mechanical Analysis}
\label{sec:analysis}
In the following paragraphs the effects of the impact of the car on the femur of
the pedestrian are investigated. First an estimation of the forces exerted on
the femur is made. Then a force diagram and bending moment diagram are calculated.
Based on these diagrams the stresses in the bone are estimated and by comparing
the stresses with experimental data of previous research the probability that
the bone will fail is estimated.

\subsection{Calculating impact force}
We start from the conservation of momentum law in the direction the car is
traveling. This assumes a perfect ellastic collision, which is a very
crude approximation of reality. Because the velocity of the pedestrian in this
direction is zero, so is his momentum. \autoref{fig:events} shows that the car
and the pedestrian cling together during the first few seconds after time of impact, so
we consider them as one system with one momentum.
\begin{equation}
	p_{C_0} + 0 = p_1
\end{equation}

% \begin{equation}
% 	\frac{m_C \cdot v_{C_0}^2}{2} + \frac{m_P \cdot v_{P_0}^2}{2}  = 
% 	\frac{m_C \cdot v_{C_1}^2}{2} + \frac{m_P \cdot v_{P_1}^2}{2}
% \end{equation}

\begin{equation}
	m_C \cdot v_{C_0} + 0 = (m_C + m_P) \cdot v_1
\end{equation}

Solve for $v_1$.
\begin{equation}
	v_1 = \frac{m_C}{m_C + m_P} \cdot v_{C_0} 
	= \frac{2400\text{ kg}}{2480\text{ kg}} \cdot 10 \text{ m/s}
	\approx 9.68 \text{ m/s}
\end{equation}

From this, we can calculate the force acting on the pedestrian.
\begin{equation}
	F' = \frac{\Delta p}{\Delta t}
\end{equation}

The equation shows that the force $F$ is also dependent on the time interval
$\Delta t$ in which the collision occurs. When two hard materials hit each
other, e.g. a club hitting a golfbal, this interval is approximately $400 \mu
s$\footnote{\url{http://www.golfswing.com.au/139}}. On the other hand, the
typical impact time - from full speed to standstill - of a car crashing into a wall is
about 100ms \cite{crash_mech}
\footnote{\url{http://auto.howstuffworks.com/car-driving-safety/accidents-hazardous-conditions/crash-test.htm}}.
This interval is larger because of crumple zones. However, because our car is
equipped with a bullbar, and because we only take into account the initial
event, we estimate $\Delta t \approx 10$ms.

Plugging this value into the equation gives:
\begin{equation}
	F' = \frac{80\text{ kg} \cdot 9.68\text{ m/s}}{0.01\text{ s}} = 77440\text{ N}
\end{equation}

We assumed a soft tissue damping factor of roughly 15\%, which allows us to
calculate the resulting foce on the femur:
\begin{equation}
	F = 0.85F' = 65824\text{ N}
\end{equation}

\subsection{Discussion on force and bending moment diagrams} 
After the mechanical system in \autoref{sec:assumptions} was decided upon, the
force and bending moment equilibrium equations were calculated in the y-direction.

\begin{equation}
	\Sigma F_y = 0: R_{Ay} + R_{By} - F = 0
\end{equation}

\begin{equation}
	\Sigma M_z \stackrel{w.r.t. B}{=} 0: L_2 F - (L_1 + L_2)R_{Ay} - M_A = 0 
\end{equation}

We did not calculate the forces in the x-direction, as these were less
relevant for our analysis.

From these equations it is clear we have a hyperstatic sytem of level 1, which
means we have three unknowns and only two equations. Therefore, this system has
to be solved using �the method of the chord�. The first step in applying this
technique is choosing two points $a$ and $b$ that are fixed in your system. Then
a point $c$ is determined which is situated between points $a$ and $b$. Point
$c$ indicates where you want to calculate the deformation in the
system due to external forces and bending moments. In this case point $c$ coincides with point $a$. Next the bending moment
diagram $M$, taking into account all external forces, is calculated. From this
bending moment diagram $M$ the reduced bending moment $M_{red}$ is deduced by dividing M by the
multiplication of the Young�s modulus and the moment of inertia of the system
($EI$). In this way the stiffness of the system is also taken into account. This
allows us to get an exact solution for the hyperstatic mechanical system.


In the next step the reduced bending moment diagram is treated as a
distributed force acting on the entire length of the beam. To calculate the equivalent force of these
distributed forces the reduced bending moment diagram is divided into triangular
parts. For each of these triangular parts an $F_e$ is calculated. By
applying a force and bending moment equilibrium on each part of the beam the
reaction force in point $c$ is determined.
The method of the chord then states that the reaction force in this point
is equal to the angle of deformation in that point.

In order to solve the hyperstatic system it has to be divided into two
subsystems: a main system (\autoref{fig:hyper2main}) and a recovery system
(\autoref{fig:hyper3recovery}). The main system is the hyperstatic system made
static again by cancelling one force or bending moment. In this case we chose to cancel
the bending moment of the clamped pelvis. The task of the recovery system is to restore
the distortion, due to cancelling the bending moment, of the line of deformation
of the main system. Therefore, a bending moment $m_a$ is added in the
recovery system. The method of the chord is applied on both systems and then the
results of the two systems are superimposed to obtain the results for the
initial hyperstatic system.

 \begin{figure}[htp]
\begin{center}
  \includegraphics[page=2,width=\textwidth]{img/hyper.pdf}
  \caption{The main system (HS).}
  \label{fig:hyper2main}
\end{center}
\end{figure}

\begin{figure}[htp]
\begin{center}
  \includegraphics[page=3,width=\textwidth]{img/hyper.pdf}
  \caption{The recovery system (hs).}
  \label{fig:hyper3recovery}
\end{center}
\end{figure}

\subsubsection{Analysis of the main system}
First, we calculate the bending moment $M$ for the main system. From this
bending moment $M$ the reduced bending moment $M_{red}$ is deduced in
\autoref{fig:hyper4}.

\begin{equation}
	M_{HS} = R_A \cdot L_1 = \frac{-L_1 \cdot L_2}{L_1 + L_2} \cdot F
\end{equation}

\begin{equation}
	M_{red} = \frac{-L_1 \cdot L_2}{L_1 + L_2} \cdot \frac{F}{EI}
\end{equation}

Based on $M_{red}$ the equivalent forces $F_e$ are calculated. Then $R_A$ (and
thus $\theta_{HS}$) is calculated.

\begin{equation}
	F_{e_1} = \frac{-L_1 L_2}{L_1+ L_2} \frac{F}{EI} \frac{L_1}{2}
\end{equation}

\begin{equation}
	F_{e_2} = \frac{-L_1 L_2}{L_1+ L_2} \frac{F}{EI} \frac{L_2}{2}
\end{equation}

\begin{equation}
	\Sigma M_z \stackrel{w.r.t. b}{=} 0: \frac{2}{3} L_2 F_{e_2} + (L_2 +
	\frac{1}{3}L_1) F_{e_1} - R_A (L_1 + L_2) = 0
\end{equation}

\begin{equation}\label{eq:RA}
	R_A = \frac{\frac{2}{3} L_2 F_{e_2} + L_2 F_{e_1} + \frac{1}{3}L_1 F_{e_1}}{L_1 +
	L_2} = \theta_{HS}
\end{equation}

\begin{equation}
	R_A = \frac{-1}{3} \frac{L_1 L_2^3 F}{(L_1 + L_2)^2 EI} - \frac{L_1^2 L_2^2
	F}{(L_1 + L_2)^2 2EI} - \frac{1}{6} \frac{L_1^3 L_2 F}{(L_1 + L_2)^2 EI} =
	\theta_{HS}
\end{equation}

\begin{figure}[htp]
\begin{center}
  \includegraphics[page=4,width=\textwidth]{img/hyper.pdf}
  \caption{Reduced bending moment diagram $M_{red}$ of the main system.}
  \label{fig:hyper4}
\end{center}
\end{figure}

\subsubsection{Analysis of the recovery system}
We repeat this exercise for the recovery system in \autoref{fig:hyper5}.
\begin{equation}
	M_{hs} = -m = -m_a
\end{equation}

\begin{equation}
	M_{red} = \frac{-m}{EI}
\end{equation}

\begin{equation}
	\Sigma M_z \stackrel{w.r.t. b}{=} 0: \frac{2}{3}(L_1 + L_2)F_e - (L_1+L_2) R_a = 0
\end{equation}

Solve for $R_a$:
\begin{equation}
	R_a = \frac{\frac{2}{3}(L_1 + L_2)F_e}{L_1 + L_2} = \frac{2}{3} F_e
\end{equation}

We also know $F_e$:
\begin{equation}
	F_e = \frac{-m}{EI} \frac{L_1 + L_2}{2}
\end{equation}

So we can rewrite $R_a$ as:
\begin{equation}
	R_a = \frac{-2}{3} \frac{m}{EI} \frac{L_1 + L_2}{2} = \theta_{hs}
\end{equation}

\begin{figure}[htp]
\begin{center}
  \includegraphics[page=5,width=\textwidth]{img/hyper.pdf}
  \caption{Reduced bending moment diagram of the recovery system.}
  \label{fig:hyper5}
\end{center}
\end{figure}

\subsubsection{Combining both systems}
To combine the two systems, the following equation must hold:
\begin{equation}
	\theta_{HS} + \theta_{hs} = 0
\end{equation}

\begin{equation}
	\frac{-1}{3} \frac{L_1 L_2^3 F}{(L_1 + L_2)^2 EI} - \frac{L_1^2 L_2^2
	F}{(L_1 + L_2)^2 2EI} - \frac{1}{6} \frac{L_1^3 L_2 F}{(L_1 + L_2)^2 EI} -
	\frac{2}{3} \frac{m}{EI} \frac{L_1 + L_2}{2} = 0
\end{equation}

Solve for m:
\begin{equation}
	m = \frac{-2 L_1 L_2^3F - 3 L_1^2 L_2^2 - L_1^3 L_2 F}{2(L_1 + L_2)^3}
\end{equation}

Using $L_1 = 0.154$m, $L_2 = 0.8$m and $F = 65824$N, we can calculate $M_A$ by
applying the superposition principle in point $c$:
\begin{equation}
	M_A = 0-m = 7814.44\text{Nm}
\end{equation}

\subsubsection{Solving the original system}
With this knowledge, we can finally solve the original system and construct the
accompanying force (\autoref{fig:hyper6}) and bending moment (\autoref{fig:hyper7})
diagrams by using the principle of superposition.

\begin{equation}
	R_{Ay} = \frac{L_2}{L_1 + L_2} F = 55 198.32 \text{N}
\end{equation}

\begin{equation}
	R_{By} = \frac{L_1}{L_1 + L_2} F = 10 625.68 \text{N}
\end{equation}

\begin{equation}
	M_A = 7814.44\text{Nm}
\end{equation}

\begin{equation}
	 M_{HS,\text{ at }F} = - \frac{L_1 L_2}{L_1 + L_2} F = - 8 500.54 \text{Nm}
\end{equation}

\begin{equation}
	 M_{hs,\text{ at }F} = \frac{L_2 M_A}{L_1 + L_2} = 6 552.98 \text{Nm}
\end{equation}

\begin{equation}
	 M_{\text{at }F} = M_{HS,\text{ at }F} + M_{hs,\text{ at }F} = - 1947,56
	 \text{Nm}
\end{equation}

\begin{figure}[htp]
\begin{center}
  \includegraphics[page=6,width=\textwidth]{img/hyper.pdf}
  \caption{Force diagram of the original system.}
  \label{fig:hyper6}
\end{center}
\end{figure}

\begin{figure}[htp]
\begin{center}
  \includegraphics[page=7,width=\textwidth]{img/hyper.pdf}
  \caption{Bending moment diagram of the original system.}
  \label{fig:hyper7}
\end{center}
\end{figure}

The forces applied on the upper part of the upper leg are definitely the
largest: they are five times larger than the forces applied on the lower parts.
Nevertheless, both distributed forces are of significant magnitude: 55,2 kN  and
10,6 kN. As they are exerted in the transverse direction on the femur, it is
very likely they will cause a fracture of the bone. All long bones have
anisotropic structure characteristics. This causes them to be weaker in the
transverse direction than in the longitudinal direction, which makes a fracture
in our scenario very likely.

\subsection{Calculating stresses in the femur}
In this section, we calculate the stresses $\sigma$ in the femur, assuming it
has not yet broken. For this, we need the applied force $F$ and the contact area
$A$. We already know $F$, so this section will focus on estimating $A$. To do
so, we make use of Hertz theory. We model both the femur and the bull bar as
cylinders, and assume they are in direct contact. This theory requires three
main assumptions to hold true:
\begin{enumerate}
  \item both materials must have a similar Young's modulus
  \item both surfaces must deform
  \item the contact area must be relatively small compared to the two bodies in
  contact
\end{enumerate}

The first assumption will hold true in our case because the Young's modulus of
the femur bone is 20 GPa and of the aluminium bull bar is 69 GPa, which is only
a factor 3.5 difference. For the second assumption we can expect that the
deformation will mostly affect the femur rather than the bull bar. The third
assumption is likely to be met as well.

Our goal is to calculate the area $A$ of the circular contact area. This
requires a contact radius $a$:
\begin{equation}
	a = \sqrt{Rd}
\end{equation}

We estimate the radius of the femur and bull bar as respectively $R_{bone} =
12.15$mm and $R_{bar} = 30.00$mm. The diameters $D_{bone}$ and $D_{bar}$ are
obviously double those values. Combining these two radii with the
formule below yields $R$:
\begin{equation}
	\frac{1}{R} = \frac{1}{R_{bone}} + \frac{1}{R_{bar}} = 115.3 \text{1/m} \Leftrightarrow R = 0.0087\text{m} 
\end{equation}

We can also calculate $d$ -- the total elastic compression at the contact
surface, measured along the line of the applied force $F$ -- using a formule
proposed by \cite{puttock1969elastic}. Note that we do not explicitely take into
account that the femur is actually composed of both cortical (shaft) and trabecular
(canal) bone. However, the Young's modulus used is that of the femur in
general. As a consequence, we use the entire diameter of the femur.

\begin{equation}
	d = \frac{(3\pi)^\frac{2}{3}}{2} \cdot F^\frac{2}{3} \cdot (v_{bone} +
	v_{bar})^\frac{2}{3} \cdot \left(\frac{1}{D_{bone}}\right)^\frac{1}{3}
\end{equation}

Herein, $v_{bone}$ and $v_{bar}$ are given by:
\begin{equation}
	v_{bone} = \frac{1 - \nu_{femur}^2}{\pi E_{femur}} = \frac{1 - 0,37^2}{\pi
	\cdot 20\text{ GPa}} = 1,3737 \cdot 10^{-11} \frac{1}{\text{Pa}}
\end{equation}
\begin{equation}
	v_{bar} = \frac{1 - \nu_{Al}^2}{\pi E_{Al}} = \frac{1 - 0.32^2}{\pi \cdot
	69\text{ GPa}} = 4.1408 \cdot 10^{-12} \frac{1}{\text{Pa}}
\end{equation}

Plug in all known variables, and we get this result:
\begin{equation}
	d = \frac{(3\pi)^\frac{2}{3}}{2} \cdot (65824\text{ N})^\frac{2}{3} \cdot
	(1,3737 \cdot 10^{-11} \frac{1}{\text{Pa}} + 4,1408 \cdot 10^{-12}
	\frac{1}{\text{Pa}})^\frac{2}{3} \cdot \left(\frac{1}{0,0243\text{
	m}}\right)^\frac{1}{3} = 0,0008585\text{ m}
\end{equation}

At last, we can find $A$ and thus $\sigma$:
\begin{equation}
	A = \pi  a^2 = \pi R d = 0,000023465 \text{ m}^2
\end{equation}

\begin{equation}
	\sigma = \frac{F}{A} = \frac{65824\text{ N}}{0,000023465 \text{ m}^2} = 28052
	\cdot 10^5\text{ Pa} \approx 2,8\text{ GPa}
\end{equation}

\subsection{Will the femur break?}

According to our calculations above, the transverse stresses at the contact
area are approximately 2,8 GPa. We compare this with the values in
\autoref{fig:femurprop}. As 2,8 GPa is much higher than 133 MPa, the maximum
compression stresses are exceeded, and thus the femur will fail. Note that we
made no distinction between cortical and trabecular bone. Because the table
only contains data for the relatively stronger cortical bone, it is safe to
assume that it will fracture regardless. 

As soon as the femur fractures, it can no longer absorb the stresses.
Instead, they are redirected to the soft tissue. However, the damage
calculations in this case are out of the scope of this report.

%TODO: F is linear in speed. Reaction forces and torques are linear in F.
% -> easy to calculate maximum speed so that femur won't break? 

\begin{figure}[htp]
\begin{center}
  \includegraphics{img/properties_femur.png}
  \caption{Properties of the femur. \cite{Ob}}
  \label{fig:femurprop}
\end{center}
\end{figure}
